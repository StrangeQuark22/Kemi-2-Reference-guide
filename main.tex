\documentclass[12pt]{article}
\usepackage[swedish]{babel}
\usepackage[version=4]{mhchem}
\usepackage{amsmath}
\usepackage[swedish]{varioref}
\usepackage{hyperref}
\hypersetup{
    colorlinks=true,
    linkcolor=black
}
\usepackage[swedish]{cleveref}
\usepackage{amsthm}
\usepackage{cancel}
\usepackage{float}
\usepackage{array}
\usepackage{enumitem}
\renewcommand{\labelitemii}{$\circ$}

\theoremstyle{definition}
\newtheorem{exm}{Exempel}

\title{Begreppssammanfattning - Kemi 2 \\ Blackebergs Gymnasium}
\author{Marcell Ziegler - NA21D}

\begin{document}
    \begin{titlepage}
        \maketitle
        \vfill
        
        \begin{center}
            \textbf{OBS!} Alla siffror/refenser som verkar vara länkar är länkar, \\ tryck gärna!
        \end{center}
    \end{titlepage}

    \tableofcontents

    \newpage

    \part{Kemisk jämvikt}
    
    En jämvikt är en kemisk reaktion som går åt båda håll med samma reaktionshastighet (lika snabbt). Detta medför att förhållandet mellan reaktanter och produkter förblir densamma. Egentligen är alla reaktioner jämvikter men vissa är så pass förskjutna åt ena hållet att de betraktas som fullständiga. Tecknet $\ce{<=>}$ används för att visa jämvikt, se följande exempel:
    \begin{equation*}
        \ce{HCl + H2O <=> H3O+ + Cl-}
    \end{equation*}

    \input{chapters/jämviktskonstant.tex}
    \setcounter{exm}{0}
    \input{chapters/förskjutning.tex}
    \setcounter{exm}{0}

    \pagebreak

    \part{Reaktionshastighet}

    Reaktionshastighet är en annan central del av denna kurs. Kortfattat är det hur snabbt en reaktion sker uttryckt i $\mathrm{\left[\frac{Molar}{Sekund} = \frac{M}{s} = \frac{mol}{dm^3 \cdot s}\right]}$. Detta ger oss formeln 
    \begin{equation*}
        v = \frac{\Delta C}{\Delta t} 
    \end{equation*}
    där $v = \text{reaktionshastighet i } \mathrm{M/s}$. Om man av någon anledning hade velat teckna en funktion hade $v = C'(t) = \frac{dC}{dt}$ gällt.
\end{document}