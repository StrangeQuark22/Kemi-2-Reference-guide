\documentclass[12pt]{article}
\usepackage[swedish]{babel}
\usepackage[version=4]{mhchem}
\usepackage{amsmath}
\title{Begreppssammanfattning - Kemi 2 \\ Blackebergs Gymnasium}
\author{Marcell Ziegler - NA21D}

\begin{document}
    \begin{titlepage}
        \maketitle
    \end{titlepage}

    \tableofcontents

    \newpage

    \part{Kemisk jämvikt}
    
    En jämvikt är en kemisk reaktion som går åt båda håll lika snabbt. Detta medför att förhållandet mellan reaktanter och produkter förblir densamma. Egentligen är alla reaktioner jämvikter men vissa är så pass förskjutna åt ena hållet att de betraktas som fullständiga. Tecknet $\ce{<=>}$ används för att visa jämvikt, se följande exempel:
    \begin{equation*}
        \ce{HCl + H2O <=> H3O+ + Cl-}
    \end{equation*}

    \input{chapters/1-jämviktskonstanten.tex}
\end{document}