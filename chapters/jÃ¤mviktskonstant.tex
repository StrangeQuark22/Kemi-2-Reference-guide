\section{Jämviktskonstanten}

Varje kemisk jämvikt har en s.k. jämviktskonstant $K$. Den beräknas enligt följande formel\footnote{Se s. 42--48 samt uppgift 3:1--3:3}:
\begin{equation*}
    \label{eq:jmvkonstant}
    K = \frac{\prod_{n=1}^{n_{prod}}[\mathrm{produkt}_n]}{\prod_{n=1}^{n_{reakt}}[\mathrm{reaktant}_n]}
\end{equation*}
alltså\ldots
\begin{equation*}
    K = \frac{[\mathrm{produkt}_1] \cdot [\mathrm{produkt}_2] \dotsm [\mathrm{produkt}_{n_{prod}}]}{[\mathrm{reaktant}_1] \cdot [\mathrm{reaktant}_2] \dotsm [\mathrm{reaktant}_{n_{reakt}}]}
\end{equation*}
där $n_{prod} = \text{antal produkter och } n_{reakt} = \text{antal reaktanter}$. $K$ visar förhållandet mellan produkterna av koncentrationerna av produkterna och reaktanterna. Detta leder även till dessa två till slutsatser:
\begin{align*}
    \text{större } K \Rightarrow \text{mindre reakt. eller mer prod. i jämförelse} \\
    \text{mindre } K \Rightarrow \text{mer reakt. eller mindre prod. i jämförelse} 
\end{align*}
\begin{exm}
    Vid jämvikt finns det $ \mathrm{0.045\,M} \ \ce{H2O},\ \mathrm{0.005\,M} \ \ce{H2} \text{ och } \mathrm{0.0025\,M} \ \ce{O2}$ i reaktionen
    \begin{equation*}
        \ce{2H2O <=> 2H2 + O2}
    \end{equation*}
    Sätter man in siffrorna får man
    \begin{equation*}
        K = \frac{\mathrm{[H_2]^2 \cdot [O_2]}}{\mathrm{[H_2O]^2}} = \frac{0.005^2\,\mathrm{M^2} \cdot 0.025\,\mathrm{M}}{0.045^2\,\mathrm{M^2}} \approx 3.09 \cdot 10^{-4} \, \mathrm{M}
    \end{equation*}
    Lägg märke till att vissa koncentrationer är upphöjda till en exponent. Denna exponent är alltid samma som ämnets koefficient i reaktionen. \\ $\ce{2H2O \rightarrow [H2O]^2}$ exempelvis.
\end{exm}

\subsection{Enheten på \textit{K}}

Den beräknas med en enhetsanalys på koncentrationerna\footnote{Se uppgift 3:4}.

\begin{exm}
    Givet situationen från ovan, sätt in enheter:
    \begin{equation*}
        K \approx 3.09 \cdot 10^{-4} \mathrm{ \left[\frac{M^2 \cdot M}{M^2} = \frac{M^{\cancelto{1}{3}}}{M^{\cancel{2}}} = M\right]}
    \end{equation*}
\end{exm}

\subsection{Räkna på \textit{K}}
\label{sec:räknak}
Du ska kunna räkna ut $K$ för en viss reaktion utifrån ett fåtal substansmängder eller koncentrationer\footnote{Se s. 48--49 samt uppgift 3:7}. Detta kan utföras genom att resonera kring de olika ämnenas koncentrationer mycket likt hur vi räknade på begränsande reaktanter i Kemi 1.

\begin{exm}
    Titta på följande tabell:
    \begin{table}[H]
        \centering
        \begin{tabular}{|c|c|>{\centering\arraybackslash}m{31.5pt}|c|} 
        \cline{2-4}
        \multicolumn{1}{l|}{} & \multicolumn{3}{c|}{$\ce{\hspace{7pt} A \hspace{7pt} + \hspace{7pt} B \hspace{2pt} <=> \hspace{2pt} AB} \hspace{15pt}$}  \\ 
        \hline
        $C_0$                 & $x$   & $x$   & $0$                       \\ 
        \hline
        $\Delta C$            & $-y$  & $-y$  & $+y$                      \\ 
        \hline
        $C_{jmv}$             & $x-y$ & $x-y$ & $y$                       \\
        \hline
        \end{tabular}
    \end{table}
    
    \noindent $C_0$ är koncentrationen från början och $C_{jmv}$ är koncentrationen vid jmv. Av detta följer att
    
    \begin{equation*}
        K = \frac{[\mathrm{AB}]}{\mathrm{[A] \cdot [B]}} = \frac{y}{(x-y)^2} \, \mathrm{\left[\frac{M}{M^2}=M^{-1}\right]}
    \end{equation*}
    
    \noindent Notera att förhållendet mellan $\Delta C$ hos de olika ämnen är densamma som deras koefficient i rekationen så följande gäller i mer komplexa fall:
    
    \begin{table}[H]
        \centering
        \begin{tabular}{|c|c|>{\centering\arraybackslash}m{40pt}|c|} 
        \cline{2-4}
        \multicolumn{1}{l|}{} & \multicolumn{3}{c|}{$\ce{\hspace{5pt} $n$A \hspace{6pt} + \hspace{13pt} B \hspace{5pt} <=> \hspace{2pt} A_nB} \hspace{15pt}$}  \\ 
        \hline
        $C_0$                 & $z$   & $x$   & $0$                       \\ 
        \hline
        $\Delta C$            & $-ny$  & $-y$  & $+y$                      \\ 
        \hline
        $C_{jmv}$             & $z-ny$ & $x-y$ & $y$                       \\
        \hline
        \end{tabular}
    \end{table}
    
    \begin{equation*}
        K = \frac{[\mathrm{A}_n\mathrm{B}]}{[\mathrm{A}]^n \cdot [\mathrm{B}]} = \frac{y}{(z-ny)^n \cdot (x-y)} \, \mathrm{\left[\frac{M}{M^2}=M^{-1}\right]}
    \end{equation*}
\end{exm}